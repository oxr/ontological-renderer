
\documentclass[12pt]{article}
\usepackage[a4paper, margin=1in]{geometry}
\usepackage{amsmath}
\usepackage{amsfonts}
\usepackage{lmodern}
\usepackage{hyperref}

\title{Ontological Renderer: Lighting Model Refinement}
\author{}
\date{}

\begin{document}
\maketitle

\section*{Lighting Model Requirements}

We want the following shading behavior:

\begin{itemize}
  \item Front-lit point ($\mathbf{n} \cdot \mathbf{l} = 1$): shade = 1.0
  \item Terminator ($\mathbf{n} \cdot \mathbf{l} = 0$): shade = 0.25
  \item Back-lit point ($\mathbf{n} \cdot \mathbf{l} = -1$): shade = 0.5
\end{itemize}

\section*{Shading Function Form}

We define:
\[
x = \mathbf{n} \cdot \mathbf{l}
\]
And use a piecewise linear function:
\[
\text{shade}(x) = a + b \cdot \max(0, x) + c \cdot \max(0, -x)
\]

Plug in values:

\begin{align*}
x = 1: &\quad \text{shade} = a + b = 1.0 \\
x = 0: &\quad \text{shade} = a = 0.25 \\
x = -1: &\quad \text{shade} = a + c = 0.5
\end{align*}

Solving:
\[
a = 0.25, \quad b = 0.75, \quad c = 0.25
\]

\section*{Final Shading Equation}

\[
\text{shade}(x) = 0.25 + 0.75 \cdot \max(0, x) + 0.25 \cdot \max(0, -x)
\]
Or directly in terms of the dot product:
\[
\text{shade} = 0.25 + 0.75 \cdot \max(0, \mathbf{n} \cdot \mathbf{l}) + 0.25 \cdot \max(0, -\mathbf{n} \cdot \mathbf{l})
\]

This ensures:
\begin{itemize}
  \item Fully lit side reaches 1.0
  \item Terminator drops to 0.25
  \item Back-lit side reflects light up to 0.5
\end{itemize}

\end{document}
